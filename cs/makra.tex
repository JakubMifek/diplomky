%%% Tento soubor obsahuje definice různých užitečných maker a prostředí %%%
%%% Další makra připisujte sem, ať nepřekáží v ostatních souborech.     %%%

%%% Makra pro definice, věty, tvrzení, příklady, ... (vyžaduje baliček amsthm)

\theoremstyle{plain}
\newtheorem{veta}{Věta}
\newtheorem{lemma}[veta]{Lemma}
\newtheorem{tvrz}[veta]{Tvrzení}

\theoremstyle{plain}
\newtheorem{definice}{Definice}

\theoremstyle{remark}
\newtheorem*{dusl}{Důsledek}
\newtheorem*{pozn}{Poznámka}
\newtheorem*{prikl}{Příklad}

%%% Prostředí pro důkazy (vyžaduje balíček bbding)

\newenvironment{dukaz}{
  \par\medskip\noindent
  \textit{Důkaz}.
}{
\newline
\rightline{\SquareCastShadowBottomRight}
}

%%% Prostředí pro sazbu kódu, případně vstupu/výstupu počítačových
%%% programů. (Vyžaduje balíček fancyvrb -- fancy verbatim)

\DefineVerbatimEnvironment{code}{Verbatim}{fontsize=\small, frame=single}

%%% Prostor reálných, resp. přirozených čísel
\newcommand{\R}{\mathbb{R}}
\newcommand{\N}{\mathbb{N}}

%%% Užitečné operátory pro statistiku a pravděpodobnost
\DeclareMathOperator{\pr}{\textsf{P}}
\DeclareMathOperator{\E}{\textsf{E}\,}
\DeclareMathOperator{\var}{\textrm{var}}
\DeclareMathOperator{\sd}{\textrm{sd}}

%%% Příkaz pro transpozici vektoru/matice
\newcommand{\T}[1]{#1^\top}

%%% Vychytávky pro matematiku
\newcommand{\goto}{\rightarrow}
\newcommand{\gotop}{\stackrel{P}{\longrightarrow}}
\newcommand{\maon}[1]{o(n^{#1})}
\newcommand{\abs}[1]{\left|{#1}\right|}
\newcommand{\dint}{\int_0^\tau\!\!\int_0^\tau}
\newcommand{\isqr}[1]{\frac{1}{\sqrt{#1}}}

%%% Vychytávky pro tabulky
\newcommand{\pulrad}[1]{\raisebox{1.5ex}[0pt]{#1}}
\newcommand{\mc}[1]{\multicolumn{1}{c}{#1}}
